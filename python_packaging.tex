\documentclass{beamer}
\usepackage[utf8]{inputenc}
%\usepackage[spanish]{babel}
\usepackage{hyperref}
\usepackage{ulem}\normalem
\usepackage{graphicx}
% \input
\parskip 10.9pt
\usetheme{AnnArbor}
\usecolortheme {rose}
\title{Python Packaging}
\author[noahfx]{Josu\'e Ortega  \\ \texttt{http://josueortega.org}}
\institute{Debian Guatemala}
\begin{document}
\begin{frame}
  \titlepage
      {\tiny
        \begin{center}
          \begin{tabular}{l@{\hspace{1em}}l}    
            licencia
            & \href{http://creativecommons.org/licenses/by-sa/3.0/}{CC BY-SA 3.0 ---
              Creative Commons Attribution-ShareAlike 3.0} \\       
          \end{tabular}
      \end{center}}
\end{frame}
\section{¿Por Qu\'e empaquetar nuestros programas hechos en Python?}
\begin{frame}{¿Por Qu\'e empaquetar nuestros programas hechos en Python?}
  \begin{itemize}
    \pause
  \item F\'acilidad de instalaci\'on (setuptools)
    \pause 
  \item F\'acilidad de compartir 
    \pause
  \item F\'acilidad de distribuci\'on (pypi, pip, easy\_install)
  \end{itemize}
\end{frame}

\section{¿Qu\'e se necesita para empaquetar?}
\begin{frame}{¿Qu\'e se necesita para empaquetar?}
  \begin{itemize}
    \pause
  \item Un programa escrito en Python :)
    \pause
  \item setuptools (Magia)
    \pause
  \item twine (opcional)
  \end{itemize}
\end{frame}

\subsection{setuptools}
\begin{frame}{setuptools}
  Es una colecci\'on de mejoras de los Python {\bf distutils} que ayudan a hacer
  mas f\'acil construir y distribuir {\em Paquetes} especialmente
  con aquellas que dependen de otros paquetes.

\end{frame}
\section{Estructura Minima}
\begin{verbatim}
    cgsolpkg/
            cgsolpkg/
            *setup.py 
            README.rst
            MANIFEST.in
            data/
\end{verbatim}
\begin{section}{setup.py}
  \begin{frame}{setup.py}
    \begin{itemize}
    \item Name
      \pause
    \item Version
      \pause
    \item Packages
      \pause
    \item Metadata
      \pause
    \item Dependencies
      \pause
    \item Package Data
      \pause
    \item Data Files
      \pause
    \item Scripts
    \end{itemize}
  \end{frame}
  \begin{section}{Archivos Opcionales}
    \begin{frame}{Archivos Opcionales}
      \begin{itemize}
      \item MANIFEST.in
        \pause
      \item README
      \end{itemize}
    \end{frame}
    \begin{section}{Desarrollando el Proyecto}
      \begin{frame}[fragile]{Desarrollando el Proyecto}
\begin{verbatim}
  #setuptools way
  python setup.py develop 
  #pip way
  pip install -e 
\end{verbatim}
      \end{frame}
      \begin{section}{Empaquetando}
        \subsection{PyPI}
        \begin{frame}{PyPI}
          \alert(Python Package Index) 
          Es el Index de paquetes Python por default para la comunidad de Python. 
          Esta abierto a todos las personas que desarrollan en Python para consumir y distribuir
          sus paquetes

          \alert https://pypi.python.org/pypi
        \end{frame}
        \begin{subsection}{Built Distributions vs Source Distributions}
          \begin{frame}{Built Distributions}
            Son distribuciones donde el manejador de paquetes solo se encarga de 
            copiar los archivos a sus diretorios de destino. 
          \end{frame}
          \begin{frame}{Source Distributions(sdist)}
            Son distribuciones (usualmente generadas con setuptools) que proveen la metadata 
            y las fuentes necesarias para instalarlas con herramientas como pip o para generar
            Built Distributions. 
          \end{frame}

          \begin{frame}[fragile]{Hora de la magia}
\begin{verbatim}
  python setup.py sdist
\end{verbatim}
          \end{frame}
          \begin{section}{Publicando en PyPI}
            \begin{frame}{Publicando en PyPI}
              \begin{itemize}
              \item Crear cuenta
              \item Subir el paquete
              \end{itemize}

            \end{frame}
          \end{document}

