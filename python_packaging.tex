\documentclass{beamer}
\usepackage[utf8]{inputenc}
\usepackage[spanish]{babel}
\usepackage{hyperref}
\usepackage{ulem}\normalem
\usepackage{graphicx}
% \input
\parskip 10.9pt
\usetheme{AnnArbor}
\usecolortheme {rose}
\title{Python Packaging}
\author[noahfx]{Josu\'e Ortega  \\ \texttt{http://josueortega.org}}
\institute{Debian Guatemala}
\begin{document}
\begin{frame}
  \titlepage
    {\tiny
    \begin{center}
      \begin{tabular}{l@{\hspace{1em}}l}    
         licencia
        & \href{http://creativecommons.org/licenses/by-sa/3.0/}{CC BY-SA 3.0 ---
          Creative Commons Attribution-ShareAlike 3.0} \\       
      \end{tabular}
    \end{center}}
\end{frame}
\section{¿Por Qu\'e empaquetar nuestros programas hechos en Python?}
\begin{frame}{¿Por Qu\'e empaquetar nuestros programas hechos en Python?}
\begin{itemize}
\pause
\item F\'acilidad de instalaci\'on (setuptools)
\pause 
\item F\'acilidad de compartir 
\pause
\item F\'acilidad de distribuci\'on (pypi, pip, easy\_install)
\end{itemize}
\end{frame}

\section{¿Qu\'e se necesita para empaquetar?}
\begin{frame}{¿Qu\'e se necesita para empaquetar?}
\begin{itemize}
\pause
\item Un programa escrito en Python :)
\pause
\item setuptools (Magia)
\pause
\item twine (opcional)
\end{itemize}
\end{frame}

\subsection{setuptools}
\begin{frame}{setuptools}
Es una colecci\'on de mejoras de los Python {\bf distutils} que ayudan a hacer
mas f\'acil construir y distribuir {\em Paquetes} especialmente
con aquellas que dependen de otros paquetes.

\end{frame}
\section{Archivos Necesarios}
\begin{frame}
\alert 
\end{frame}
\end{document}

