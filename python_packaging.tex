\documentclass{beamer}
\usepackage[utf8]{inputenc}
%\usepackage[spanish]{babel}
\usepackage{hyperref}
\usepackage{ulem}\normalem
\usepackage{graphicx}
% \input
\parskip 10.9pt
\usetheme{AnnArbor}
\usecolortheme {rose}
\title{Python Packaging}
\author[noahfx]{Josu\'e Ortega  \\ \texttt{http://josueortega.org}}
\institute{Debian Guatemala}
\begin{document}
\begin{frame}
  \titlepage
    {\tiny
    \begin{center}
      \begin{tabular}{l@{\hspace{1em}}l}    
         licencia
        & \href{http://creativecommons.org/licenses/by-sa/3.0/}{CC BY-SA 3.0 ---
          Creative Commons Attribution-ShareAlike 3.0} \\       
      \end{tabular}
    \end{center}}
\end{frame}
\section{¿Por Qu\'e empaquetar nuestros programas hechos en Python?}
\begin{frame}{¿Por Qu\'e empaquetar nuestros programas hechos en Python?}
\begin{itemize}
\pause
\item F\'acilidad de instalaci\'on (setuptools)
\pause 
\item F\'acilidad de compartir 
\pause
\item F\'acilidad de distribuci\'on (pypi, pip, easy\_install)
\end{itemize}
\end{frame}

\section{¿Qu\'e se necesita para empaquetar?}
\begin{frame}{¿Qu\'e se necesita para empaquetar?}
\begin{itemize}
\pause
\item Un programa escrito en Python :)
\pause
\item setuptools (Magia)
\pause
\item twine (opcional)
\end{itemize}
\end{frame}

\subsection{setuptools}
\begin{frame}{setuptools}
Es una colecci\'on de mejoras de los Python {\bf distutils} que ayudan a hacer
mas f\'acil construir y distribuir {\em Paquetes} especialmente
con aquellas que dependen de otros paquetes.

\end{frame}
\section{Estructura Minima}
\begin{verbatim}
    cgsolpkg/
            cgsolpkg/
            *setup.py 
            README.rst
            MANIFEST.in
            data/
\end{verbatim}
\begin{section}{setup.py}
\begin{frame}{setup.py}
\begin{verbatim}
from setuptools import setup, find_packages  # Always prefer setuptools over distutils
from codecs import open  # To use a consistent encoding
from os import path

with open(path.join(here, 'DESCRIPTION.rst'), encoding='utf-8') as f:
    long_description = f.read()
setup(
    name='sample',

    version='1.2.0',

    description='A sample Python project for CGSOL',
    long_description=long_description,

    url='josueortega@debian.org.gt',

    author='The Python Packaging Authority',
    author_email='joseortega@debian.org.gt',

    license='MIT',


    classifiers=[
        'Development Status :: 3 - Alpha',
        'Intended Audience :: Developers',
        'Topic :: Software Development :: Build Tools',
        'License :: OSI Approved :: MIT License',

        'Programming Language :: Python :: 2',
        'Programming Language :: Python :: 2.6',
        'Programming Language :: Python :: 2.7',
        'Programming Language :: Python :: 3',
        'Programming Language :: Python :: 3.2',
        'Programming Language :: Python :: 3.3',
        'Programming Language :: Python :: 3.4',
    ],


    keywords='sample setuptools development',

    packages=find_packages(exclude=['contrib', 'docs', 'tests*']),

    install_requires=['peppercorn'],

    extras_require = {
        'dev': ['check-manifest'],
        'test': ['coverage'],
    },

    # If using Python 2.6 or less, then these
    # have to be included in MANIFEST.in as well.
    package_data={
        'sample': ['package_data.dat'],
    },
    data_files=[('my_data', ['data/data_file'])],
    entry_points={
        'console_scripts': [
            'sample=sample:main',
        ],
    },
)
\end{verbatim}
\end{frame}
\begin{section}{Archivos Opcionales}
\begin{frame}{Archivos Opcionales}
\begin{itemize}
  \item MANIFEST.in
  \item README
\end{itemize}
\end{frame}
\end{document}

